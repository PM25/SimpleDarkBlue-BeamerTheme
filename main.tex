%----------------------------------------------------------------------------------------
%	PACKAGES AND THEMES
%----------------------------------------------------------------------------------------
\documentclass[aspectratio=169,xcolor=dvipsnames]{beamer}

\mode<presentation> {
    \usetheme{Madrid}
    %\setbeamertemplate{footline} % To remove the footer line in all slides uncomment this line
    %\setbeamertemplate{footline}[page number] % To replace the footer line in all slides with a simple slide count uncomment this line
    \setbeamertemplate{navigation symbols}{} % To remove the navigation symbols from the bottom of all slides uncomment this line
    
    \definecolor{InvisibleRed}{rgb}{0.92, 0.9, 0.9}
    \definecolor{InvisibleGreen}{rgb}{0.9, 0.92, 0.9}
    \definecolor{InvisibleBlue}{rgb}{0.9, 0.9, 0.92}

    \definecolor{LightBlue}{rgb}{0.4, 0.55, 0.65}
    
    \definecolor{MediumRed}{rgb}{0.92549, 0.34509, 0.34509}
    \definecolor{MediumGreen}{rgb}{0.36862, 0.66666, 0.65882}
    \definecolor{MediumBlue}{rgb}{0.01176, 0.31372, 0.43529}
    
    \definecolor{DarkBlue}{rgb}{0.04706, 0.13725, 0.26667} 
    \usecolortheme[named=DarkBlue]{structure}
    % \useoutertheme{miniframes} % Alternatively: miniframes, infolines, split
    \setbeamercolor{palette primary}{bg=DarkBlue,fg=white}
    \setbeamercolor{palette secondary}{bg=MediumBlue,fg=white}
    \setbeamercolor{palette tertiary}{bg=LightBlue,fg=white}
    \setbeamercolor{block title}{bg=MediumBlue}
    \setbeamercolor{block body}{bg=InvisibleBlue}
    \setbeamercolor{block title example}{bg=MediumGreen}
    \setbeamercolor{block body example}{bg=InvisibleGreen}
    \setbeamercolor{block title alerted}{bg=MediumRed}
    \setbeamercolor{block body alerted}{bg=InvisibleRed}
    
    \useinnertheme{circles}
}

\usepackage{hyperref}
\usepackage{graphicx} % Allows including images
\usepackage{booktabs} % Allows the use of \toprule, \midrule and \bottomrule in tables

% \setbeamertemplate{section in toc}[square]
% \setbeamertemplate{subsection in toc}[square]

% \setbeamertemplate{itemize items}[square]
% \setbeamertemplate{itemize subitem}[triangle]


%----------------------------------------------------------------------------------------
%	TITLE PAGE
%----------------------------------------------------------------------------------------

\title[short title]{\LARGE{Title}} % The short title appears at the bottom of every slide, the full title is only on the title page
\subtitle{Subtitle}

\author[Pin-Yen] {
    \LARGE {Pin-Yen Huang} \\
    \medskip
    \footnotesize{\href{mailto:pyhuang97@gmail.com}{pyhuang97@gmail.com}} % Your email address
    \vspace{-1ex}
}

\institute[NTU] % Your institution as it will appear on the bottom of every slide, may be shorthand to save space
{
    Department of Computer Science and Information Engineering \\ 
    National Taiwan University % Your institution for the title page
    \vskip 3pt
}
\date{\today} % Date, can be changed to a custom date

\begin{document}

\begin{frame}
    \titlepage % Print the title page as the first slide
\end{frame}

\begin{frame}
    \frametitle{Overview} % Table of contents slide, comment this block out to remove it
    \tableofcontents % Throughout your presentation, if you choose to use \section{} and \subsection{} commands, these will automatically be printed on this slide as an overview of your presentation
\end{frame}


%----------------------------------------------------------------------------------------
%	PRESENTATION SLIDES
%----------------------------------------------------------------------------------------

%------------------------------------------------
\section{First Section}
%------------------------------------------------

\begin{frame}{Bullet Points}
    \begin{itemize}
        \item Lorem ipsum dolor sit amet, consectetur adipiscing elit
        \item Aliquam blandit faucibus nisi, sit amet dapibus enim tempus eu
        \item Nulla commodo, erat quis gravida posuere, elit lacus lobortis est, quis porttitor odio mauris at libero
        \item Nam cursus est eget velit posuere pellentesque
        \item Vestibulum faucibus velit a augue condimentum quis convallis nulla gravida
    \end{itemize}
\end{frame}

%------------------------------------------------

\begin{frame}
    \frametitle{Blocks of Highlighted Text}
    In this slide, some important text will be \alert{highlighted} because it's important. Please, don't abuse it.

    \begin{block}{Block}
        Sample text
    \end{block}

    \begin{alertblock}{Alertblock}
        Sample text in red box
    \end{alertblock}

    \begin{examples}
        Sample text in green box. The title of the block is ``Examples".
    \end{examples}
\end{frame}

%------------------------------------------------

\begin{frame}
    \frametitle{Multiple Columns}
    \begin{columns}[c] % The "c" option specifies centered vertical alignment while the "t" option is used for top vertical alignment

        \column{.45\textwidth} % Left column and width
        \textbf{Heading}
        \begin{enumerate}
            \item Statement
            \item Explanation
            \item Example
        \end{enumerate}

        \column{.5\textwidth} % Right column and width
        Lorem ipsum dolor sit amet, consectetur adipiscing elit. Integer lectus nisl, ultricies in feugiat rutrum, porttitor sit amet augue. Aliquam ut tortor mauris. Sed volutpat ante purus, quis accumsan dolor.

    \end{columns}
\end{frame}

%------------------------------------------------
\section{Second Section}
%------------------------------------------------

\begin{frame}
    \frametitle{Table}
    \begin{table}
        \begin{tabular}{l l l}
            \toprule
            \textbf{Treatments} & \textbf{Response 1} & \textbf{Response 2} \\
            \midrule
            Treatment 1         & 0.0003262           & 0.562               \\
            Treatment 2         & 0.0015681           & 0.910               \\
            Treatment 3         & 0.0009271           & 0.296               \\
            \bottomrule
        \end{tabular}
        \caption{Table caption}
    \end{table}
\end{frame}

%------------------------------------------------

\begin{frame}
    \frametitle{Theorem}
    \begin{theorem}[Mass--energy equivalence]
        $E = mc^2$
    \end{theorem}
\end{frame}

%------------------------------------------------

\begin{frame}
    \frametitle{Figure}
    Uncomment the code on this slide to include your own image from the same directory as the template .TeX file.
    %\begin{figure}
    %\includegraphics[width=0.8\linewidth]{test}
    %\end{figure}
\end{frame}

%------------------------------------------------

\begin{frame}[fragile] % Need to use the fragile option when verbatim is used in the slide
    \frametitle{Citation}
    An example of the \verb|\cite| command to cite within the presentation:\\~

    This statement requires citation \cite{p1}.
\end{frame}

%------------------------------------------------

\begin{frame}
    \frametitle{References}
    \footnotesize{
        \begin{thebibliography}{99} % Beamer does not support BibTeX so references must be inserted manually as below
            \bibitem[Smith, 2012]{p1} John Smith (2012)
            \newblock Title of the publication
            \newblock \emph{Journal Name} 12(3), 45 -- 678.
        \end{thebibliography}
    }
\end{frame}

%------------------------------------------------

\begin{frame}
    \Huge{\centerline{The End}}
\end{frame}

%----------------------------------------------------------------------------------------

\end{document}